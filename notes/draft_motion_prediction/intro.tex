%
% File acl2015.tex
%
% Contact: car@ir.hit.edu.cn, gdzhou@suda.edu.cn
%%
%% Based on the style files for ACL-2014, which were, in turn,
%% Based on the style files for ACL-2013, which were, in turn,
%% Based on the style files for ACL-2012, which were, in turn,
%% based on the style files for ACL-2011, which were, in turn, 
%% based on the style files for ACL-2010, which were, in turn, 
%% based on the style files for ACL-IJCNLP-2009, which were, in turn,
%% based on the style files for EACL-2009 and IJCNLP-2008...

%% Based on the style files for EACL 2006 by 
%%e.agirre@ehu.es or Sergi.Balari@uab.es
%% and that of ACL 08 by Joakim Nivre and Noah Smith

\documentclass[11pt]{article}
\usepackage{acl2015}
\usepackage{times}
\usepackage{url}
\usepackage{latexsym}

%\setlength\titlebox{5cm}

% You can expand the titlebox if you need extra space
% to show all the authors. Please do not make the titlebox
% smaller than 5cm (the original size); we will check this
% in the camera-ready version and ask you to change it back.


\title{A Safety-First End-to-end Neural Motion Planning Model for Autonomous Driving}

\author{Nguyen T.T. Nguyen \\
  Advanced Intelligent System Laboratory \\
  Graduate School of Information Science and Engineering \\
  Ritsumeikan University, Shiga, Japan \\
  {\tt nguyen.aislab@gmail.com}\\}

\date{}

\begin{document}
\maketitle
\begin{abstract}
  This document contains the instructions for preparing a camera-ready
  manuscript for the proceedings of ACL-2015. The document itself
  conforms to its own specifications, and is therefore an example of
  what your manuscript should look like. These instructions should be
  used for both papers submitted for review and for final versions of
  accepted papers.  Authors are asked to conform to all the directions
  reported in this document.
\end{abstract}

\section{Introduction}

The last decade has witnessed a boom in term of research efforts, both in
institutes and corporate sector, pouring into the autonomous driving (AD) technology.
The hype is reasonable because of the huge potential of reducing accidents caused
by human factors, improving traffic flow, and providing more freedom in transportation for
disabled people, all of which this technology is expected to deliver once completed
in the future.

To control the vehicle safely, the system is required to be capable of perceive, predict,
and control the vehicle  in a highly dynamic traffic environment without human intervention.
In order to achieve that goal, although there are different approaches, AD systems in general
all shared a common pipeline started with creating progressively a representation for the
environment surrounding said vehicle from various sensors from the driver, and ended with
emitting control signals to actuators to perform the driving task autonomously based on initial
route planning inputted by the human driver. Classified as a primary safety-critical software,
AD system must strictly account for the uncertainty in both familliar and unfamilliar contexts.
In other words, the ability of measuring and understanding the uncertainty in operation is one
of the most important features for the self-driving automobile to ensure the feasibility of the
technology.
\end{document}